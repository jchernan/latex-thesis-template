\chapter{Math} \label{math}

This chapter shows different ways of including math content. It also show some \texttt{subsection} examples.

\section{Simple math content}

This paragraph has math variables inline with the sentences. If $f$ is the focal length of a camera,
and $m_x$, $m_y$ are the ratios of pixel width and pixel height per unit distance, respectively, then 
$(f_x, f_y)$ represents the focal length expressed in units of 
horizontal and vertical pixels.

\subsection{Using \texttt{align} context}

A math formula can also be included on its own line:
\begin{align}
(f_x, f_y) = ( f m_x , f m_y)
\end{align}

\section{More complex math content}

This section includes more paragraphs with more examples, including math formulas with labels and
references to those labels.

\subsection{Examples}

Furthermore, $(x_0, y_0)$ represents the principal point of the camera. Assuming that the skew coefficient 
between the $x$ and $y$ axes is zero, the intrinsic matrix of this camera is given by \ref{intrinsicmatrix}:
\begin{align} \label{intrinsicmatrix}
\mathbf{K} = \begin{bmatrix}
	f_x & 0    & x_0 \\
	0    & f_y & y_0 \\
	0    & 0    & 1      \\
\end{bmatrix}
\end{align}

The calibration process also provides a rotation matrix and a translation vector. These parameters describe
the transformation between the world's coordinate system and the camera's coordinate system. If 
$\mathbf{R}$ denotes the $3 \times 3$ rotation matrix, and $\mathbf{t}$ denotes the $3 \times 1$ 
translation vector, the extrinsic matrix of this camera is given by the $3 \times 4$ matrix 
\begin{align} \label{extrinsicmatrix}
\begin{bmatrix} \mathbf{R} & \mathbf{t} \\ \end{bmatrix}
\end{align}

Therefore, the synchronized calibration algorithm outputs two intrinsic matrices, $\mathbf{K}_d$ and 
$\mathbf{K}_c$, and two extrinsic matrices, $[ \mathbf{R}_d ~ \mathbf{t}_d ]$ and 
$[ \mathbf{R}_c ~ \mathbf{t}_c ]$, where the $d$ and $c$ subscripts are used to differentiate between the 
depth camera's parameters and the color camera's parameters. While the intrinsic matrices contain 
information about the internals of the camera, the extrinsic matrices hold information about how the 
cameras are positioned in space. This information is used in the next section to compute the relative 
transformation between both cameras. 

When computing the relative transformation between the cameras, the direction of the transformation is 
chosen to be from the depth camera to the color camera. As discussed in Section \ref{cameras}, the field
of view of the depth camera is within the field of view of the color camera. Therefore, every point in the depth
image will have a corresponding point in the color image, but not necessarily vice versa.

If $\mathbf{P} = (X, Y, Z)^T$ is a point in world coordinates, the position of $\mathbf{P}$ in the depth 
camera's coordinate system is given by $\mathbf{q}_d$. Similarly, the position of $\mathbf{P}$ in the color 
camera's coordinate system is given by $\mathbf{q}_c$. The points $\mathbf{q}_d$ and $\mathbf{q}_c$ 
can be expressed in terms of the cameras' extrinsic parameters by equations \eqref{qd} and \eqref{qc}, 
respectively.
\begin{align}
\mathbf{q}_d = \mathbf{R}_d \mathbf{P} + \mathbf{t}_d	\label{qd} \\
\mathbf{q}_c = \mathbf{R}_c \mathbf{P} + \mathbf{t}_c  	\label{qc}
\end{align}

Considering now the image of $\mathbf{P}$ in the depth image as having coordinates $(x_d, y_d)$, this
point can be expressed in homogeneous coordinates as $\mathbf{p}_d = (w x_d, w y_d, w )^T$, for some 
constant $w$. Using the depth camera's intrinsic parameters, $\mathbf{p}_d$ can be expressed by the 
equation:
\begin{align}
	\mathbf{p}_d = \mathbf{K}_d \mathbf{q}_d \label{pd}
\end{align}

This automatically reveals another expression for $\mathbf{q}_d$:
\begin{align}
	\mathbf{q}_d  = \mathbf{K}_{d}^{-1} \mathbf{p}_d \label{qd2}
\end{align}

Combining the two expressions for $\mathbf{q}_d$ (equations \eqref{qd} and \eqref{qd2}), and solving for 
$\mathbf{P}$ gives an equation for point $\mathbf{P}$:
\begin{align}
	\mathbf{K}_{d}^{-1} \mathbf{p}_d &= \mathbf{R}_d \mathbf{P} + \mathbf{t}_d \nonumber \\
	\mathbf{R}_d \mathbf{P} &= \mathbf{K}_{d}^{-1} \mathbf{p}_d - \mathbf{t}_d  \nonumber \\
	\mathbf{P} &= \mathbf{R}_{d}^{-1} \mathbf{K}_{d}^{-1} \mathbf{p}_d - \mathbf{R}_{d}^{-1} \mathbf{t}_d  
\end{align}

This expression for $\mathbf{P}$ can be substituted in equation \eqref{qc} to get a new expression for 
$\mathbf{q}_c$:
\begin{align}
	\mathbf{q}_c &= \mathbf{R}_c (\mathbf{R}_{d}^{-1} \mathbf{K}_{d}^{-1} \mathbf{p}_d 
					- \mathbf{R}_{d}^{-1} \mathbf{t}_d  ) + \mathbf{t}_c \nonumber \\
			    & =  \mathbf{R}_c \mathbf{R}_{d}^{-1} \mathbf{K}_{d}^{-1} \mathbf{p}_d
			    		 - \mathbf{R}_c \mathbf{R}_{d}^{-1} \mathbf{t}_d + \mathbf{t}_c \label{qc2}
\end{align}

Using equation \eqref{qd2}, equation \eqref{qc2} simplifies to:
\begin{align}
	\mathbf{q}_c &=  (\mathbf{R}_c \mathbf{R}_{d}^{-1}) ~ \mathbf{q}_{d}  
		+ (\mathbf{t}_c - \mathbf{R}_c \mathbf{R}_{d}^{-1} \mathbf{t}_d) \label{qc3}
\end{align}

Equation \eqref{qc3} reveals how the world points in the depth camera's coordinate system are related to 
the world points in the color camera's coordinate system. As seen from the equation, this
transformation is given in terms of the cameras' extrinsic parameters. Therefore, the relative transformation
between the depth and color cameras is defined by the rotation matrix in equation \eqref{relativerotation}
and the translation vector in equation \eqref{relativetranslation}.
\begin{align}
	\mathbf{R}_r = \mathbf{R}_c \mathbf{R}_{d}^{-1} \label{relativerotation} \\
	\mathbf{t}_r = \mathbf{t}_c - \mathbf{R}_r \mathbf{t}_d \label{relativetranslation}
\end{align}

The fusion algorithm must convert every point $(x_d, y_d)$ in the depth image into a point $(x_c, y_c)$ in 
the color image. This is achieved by first computing the world point $\mathbf{q}_d$ from the depth image
point $(x_d, y_d)$. Then, $\mathbf{q}_d$ is transformed into $\mathbf{q}_c$ using the results from Section 
\ref{relativetransformation}. Finally, world point $\mathbf{q}_c$ is converted into a color image point 
$(x_c, y_c)$.

If $(f_x, f_y)$ and $(x_0, y_0)$ are the focal length and principal point of the depth camera, respectively, 
the world point $\mathbf{q}_d = (X, Y, Z)^T$ can be related to the image point $(x_d, y_d)$ through the 
perspective projection equations \eqref{xd} and \eqref{yd}. 
\begin{align}
	x_d &= f_x \frac{X}{Z} + x_0 \label{xd} \\
	y_d &= f_y \frac{Y}{Z} + y_0 \label{yd} 
\end{align}
	
The depth camera provides the $z$-component of $\mathbf{q}_d$.\footnote{The depth camera can actually 
provide all components as discussed in Section \ref{cameras}. However, the noise in these measurements is 
high and recomputing the $x$ and $y$ components delivers better results.}
The $x$ and $y$ components are computed by solving equations \eqref{xd} and \eqref{yd} for $X$ and 
$Y$, respectively:
\begin{align}
	X = \frac{Z}{f_x} (x_d - x_0) \label{X} \\
	Y = \frac{Z}{f_y} (y_d - y_0) \label{Y}
\end{align}

Using the color camera's intrinsic parameters, $\mathbf{p}_c$ can be expressed by the equation
\begin{align}
	\mathbf{p}_c = \mathbf{K}_c \mathbf{q}_c \label{pc}
\end{align}

Furthermore, equation \eqref{qc3} gives an expression for $\mathbf{q}_c$. Therefore, combining 
\eqref{pc} and \eqref{qc3} results in a new equation for $\mathbf{p}_c$:
\begin{align}
	\mathbf{p}_c &=  \mathbf{K}_c (\mathbf{R}_r \mathbf{q}_{d}  + \mathbf{t}_r) \label{pc2}
\end{align}

By expressing $\mathbf{q}_{d}$ in homogeneous coordinates (that is, $\mathbf{q}_{d}^{'} = (X, Y, Z, 1)^T$), 
equation \eqref{pc2} can be rewritten as 
\begin{align}
	\mathbf{p}_c &=  \mathbf{K}_c [\mathbf{R}_r  ~  \mathbf{t}_r] \mathbf{q}_{d}^{'} \label{pc3}
\end{align}

The image coordinates $(x_c, y_c)$ are obtained by dividing the first and second components of 
$\mathbf{p}_c$ by its third component. That is, if $\mathbf{p}_c = (x,y,z)^T$, then 
\begin{align}
	(x_c, y_c) = \Bigg( \frac{x}{z} , \frac{y}{z} \Bigg) \label{xcyc}
\end{align}